%% Generated by Sphinx.
\def\sphinxdocclass{report}
\documentclass[letterpaper,10pt,english]{sphinxmanual}
\ifdefined\pdfpxdimen
   \let\sphinxpxdimen\pdfpxdimen\else\newdimen\sphinxpxdimen
\fi \sphinxpxdimen=.75bp\relax
\ifdefined\pdfimageresolution
    \pdfimageresolution= \numexpr \dimexpr1in\relax/\sphinxpxdimen\relax
\fi
%% let collapsible pdf bookmarks panel have high depth per default
\PassOptionsToPackage{bookmarksdepth=5}{hyperref}

\PassOptionsToPackage{booktabs}{sphinx}
\PassOptionsToPackage{colorrows}{sphinx}

\PassOptionsToPackage{warn}{textcomp}
\usepackage[utf8]{inputenc}
\ifdefined\DeclareUnicodeCharacter
% support both utf8 and utf8x syntaxes
  \ifdefined\DeclareUnicodeCharacterAsOptional
    \def\sphinxDUC#1{\DeclareUnicodeCharacter{"#1}}
  \else
    \let\sphinxDUC\DeclareUnicodeCharacter
  \fi
  \sphinxDUC{00A0}{\nobreakspace}
  \sphinxDUC{2500}{\sphinxunichar{2500}}
  \sphinxDUC{2502}{\sphinxunichar{2502}}
  \sphinxDUC{2514}{\sphinxunichar{2514}}
  \sphinxDUC{251C}{\sphinxunichar{251C}}
  \sphinxDUC{2572}{\textbackslash}
\fi
\usepackage{cmap}
\usepackage[T1]{fontenc}
\usepackage{amsmath,amssymb,amstext}
\usepackage{babel}



\usepackage{tgtermes}
\usepackage{tgheros}
\renewcommand{\ttdefault}{txtt}



\usepackage[Bjarne]{fncychap}
\usepackage{sphinx}

\fvset{fontsize=auto}
\usepackage{geometry}


% Include hyperref last.
\usepackage{hyperref}
% Fix anchor placement for figures with captions.
\usepackage{hypcap}% it must be loaded after hyperref.
% Set up styles of URL: it should be placed after hyperref.
\urlstyle{same}


\usepackage{sphinxmessages}
\setcounter{tocdepth}{1}



\title{Radii Documentation}
\date{Dec 05, 2023}
\release{0.37}
\author{Gereon Sievi}
\newcommand{\sphinxlogo}{\vbox{}}
\renewcommand{\releasename}{Release}
\makeindex
\begin{document}

\ifdefined\shorthandoff
  \ifnum\catcode`\=\string=\active\shorthandoff{=}\fi
  \ifnum\catcode`\"=\active\shorthandoff{"}\fi
\fi

\pagestyle{empty}
\sphinxmaketitle
\pagestyle{plain}
\sphinxtableofcontents
\pagestyle{normal}
\phantomsection\label{\detokenize{index::doc}}
\noindent\sphinxincludegraphics{{Radii_logo}.png}



\sphinxAtStartPar
\sphinxhref{https://radii.info/}{RADii} a Cloud platform where you can publish (upload) your
3D Models, share, explore and collaborate on them.

\sphinxAtStartPar
RADii consists of two parts, a Rhino Grasshopper plugin to publish (upload) 3D Models and Viewers for different devices (for PC/Mac, VR Glasses or mobile devices) to receive and explore the models.

\sphinxAtStartPar
RADii is being developed by Thomas Lee since 2020, in 2022 he started a collaboration with Gramazio \& Kohler Research at ETHZ for further development of the tool in the context of the Immersive Design Studio.
More information, updates and the program can be found at \sphinxhref{https://radii.info/}{radii.info}.
Videos about the current development as well as new features are available on the \sphinxhref{https://www.youtube.com/channel/UCfOGfaqPczXAGTpFDPm8XsA}{Arctica Youtube Page}
This project is still under development, should you find bugs or out of date documentation please notify us on \sphinxhref{https://github.com/Archtica/RADii/issues/}{Github}


\chapter{Download Links}
\label{\detokenize{index:download-links}}
\sphinxAtStartPar
Grashopper Plugin:
\begin{itemize}
\item {} 
\sphinxAtStartPar
\sphinxhref{https://radii.info/download/plugin/RADii.zip}{Radii Win/Mac}

\item {} 
\sphinxAtStartPar
\sphinxhref{https://radii.info/download/plugin/RADiiCapture.zip}{Radii Capture Win/Mac}

\end{itemize}

\sphinxAtStartPar
PC Viewers
\begin{itemize}
\item {} 
\sphinxAtStartPar
\sphinxhref{https://radii.info/download/standard/RADii\%20Viewer\%20Setup.zip}{Windows 10+}

\item {} 
\sphinxAtStartPar
\sphinxhref{https://apps.apple.com/app/id1505325031}{macOS X}

\end{itemize}

\sphinxAtStartPar
Further viewers for VR, Mobile, Looking Glass and the Webviewer can be found at \sphinxhref{https://radii.info/}{radii.info}.


\chapter{Table of Contents}
\label{\detokenize{index:table-of-contents}}
\sphinxstepscope


\section{Setup}
\label{\detokenize{tutorial/Setup/install_setup:setup}}\label{\detokenize{tutorial/Setup/install_setup::doc}}

\subsection{Viewer PC/Mac}
\label{\detokenize{tutorial/Setup/install_setup:viewer-pc-mac}}
\sphinxAtStartPar
To explore models in Radii you need a viewer, this is a program that enables you to join a server (we call them channels) and download the models.
Viewers are available for a number of different operating systems such as Windows/Mac, IPhone/Android and Oculus.
The viewers are generally the same but have some differences, usually due to computing power.

\noindent\sphinxincludegraphics{{Radii_Info_Downloads_Standard_viewer}.png}
\begin{enumerate}
\sphinxsetlistlabels{\arabic}{enumi}{enumii}{}{.}%
\item {} 
\sphinxAtStartPar
\sphinxstylestrong{Download} the latest Radii Viewer from \sphinxurl{https://radii.info/}

\item {} \begin{description}
\sphinxlineitem{\sphinxstylestrong{Register} in the user panel \sphinxstylestrong{and confirm} your email}\begin{itemize}
\item {} 
\sphinxAtStartPar
it is not necessary to set a domain name

\end{itemize}

\end{description}

\item {} 
\sphinxAtStartPar
\sphinxstylestrong{Install} the file

\item {} 
\sphinxAtStartPar
\sphinxstylestrong{Start} RADii Viewer

\end{enumerate}

\noindent\sphinxincludegraphics{{Menu_connect_blank}.png}

\sphinxAtStartPar
\sphinxstylestrong{Congratulations} you have installed Radii. On how to use Radii consult the Quick Guide and the Viewer documentation


\subsection{Grashopper Plugin}
\label{\detokenize{tutorial/Setup/install_setup:grashopper-plugin}}
\sphinxAtStartPar
The plugin enables you to publish (send) 3D models and other geometry.
Publishing works similar to a Radio station, geometry can be received by others as long as they are connected
to the same channel as the sender.
\begin{enumerate}
\sphinxsetlistlabels{\arabic}{enumi}{enumii}{}{.}%
\item {} 
\sphinxAtStartPar
\sphinxstylestrong{Download} the latest Radii Plugin from \sphinxurl{https://radii.info/}

\item {} \begin{description}
\sphinxlineitem{If you did not already: \sphinxstylestrong{Register} in the user panel \sphinxstylestrong{and confirm} your email}\begin{itemize}
\item {} 
\sphinxAtStartPar
it is not necessary to set a domain name

\end{itemize}

\end{description}

\item {} 
\sphinxAtStartPar
\sphinxstylestrong{Unpack} the .Zip file

\item {} 
\sphinxAtStartPar
\sphinxstylestrong{Drag \& drop} the Radii.gha file it into the Rhino Grashopper window, you open it by typing the “grashopper” command in rhino

\item {} 
\sphinxAtStartPar
\sphinxstylestrong{Check} if the install was successful, it should be visible in one your tabs as shown below.

\end{enumerate}

\noindent\sphinxincludegraphics{{Grashopper_Blank_install}.png}

\sphinxAtStartPar
\sphinxstylestrong{Congratulations !} you have installed Grashopper Radii. On how to publish consult the Quick Guide and the Radii Grashopper documentation


\subsection{Oculus}
\label{\detokenize{tutorial/Setup/install_setup:oculus}}
\sphinxAtStartPar
Please be aware that the Oculus and its environments are subject to frequent changes, the following tutorial might be out of date or diverge slightly from the current state. We are hoping to release a viewer through the app store at a later stage.
\begin{enumerate}
\sphinxsetlistlabels{\arabic}{enumi}{enumii}{}{.}%
\item {} 
\sphinxAtStartPar
Download the advanced version of sidequest from \sphinxurl{https://sidequestvr.com/} and install it

\item {} 
\sphinxAtStartPar
Download the .apk file from Radii.info in the Download section

\item {} 
\sphinxAtStartPar
Set your Oculus account to developer mode (you have to be part of a developer organization), log in to your account on your device and enable Settings \textgreater{} System \textgreater{} Developer, and then turn on the USB Connection Dialog option.

\item {} 
\sphinxAtStartPar
Connect the Oculus to your device and start side on the later

\item {} 
\sphinxAtStartPar
Allow for debugging on your Oculus

\item {} 
\sphinxAtStartPar
On sidequest select the install .apk file and install

\end{enumerate}

\sphinxAtStartPar
To find the Radii Viewer app on Oculus:
\begin{enumerate}
\sphinxsetlistlabels{\arabic}{enumi}{enumii}{}{.}%
\item {} 
\sphinxAtStartPar
Select the app library

\item {} 
\sphinxAtStartPar
Filter apps for unknown sources

\item {} 
\sphinxAtStartPar
Start Radii Viewer VR

\end{enumerate}

\sphinxAtStartPar
Congratulations you have installed Radii on your Vr device


\subsection{Technical Prerequisites}
\label{\detokenize{tutorial/Setup/install_setup:technical-prerequisites}}
\sphinxAtStartPar
To use Radii a Pc with an internet connection and Windows or Mac operating system is needed.
The computing and ram capacity can vary greatly depending on the size and detail of the 3D model.
The Rhino Grashopper plugin is generally very efficient and can run on older machines.
The Radii Viewer for PC as a live renderer need more computing power. It can be necessary to run Rhino on one machine and the viewer on another for
a seamless use.

\sphinxAtStartPar
We advice for an absolute minimum of 4GB of RAM, better would be 8GB that is current standard, or more.
The deciding technical factor for the viewer is the strength of the used graphic car

\sphinxAtStartPar
During our tests, on a range of devices, we have observed that:
\begin{itemize}
\item {} 
\sphinxAtStartPar
simple and small models can be viewed and streamed in parallel on a medium strong laptop

\item {} 
\sphinxAtStartPar
for bigger models we advice for a strong graphic card and at least 16 GB of RAM for the viewer and potentially the use of a second device for publishing with Rhino Grashopper

\end{itemize}

\sphinxAtStartPar
Server Storage:

\sphinxAtStartPar
By creating an account you get a limited amount of server space that you can use. If your models
are bigger you need to apply for more.
As member of an affiliated organization your will automatically recognized if you register with your organizations email address.


\subsection{Preparation, time and infrastructure}
\label{\detokenize{tutorial/Setup/install_setup:preparation-time-and-infrastructure}}
\sphinxAtStartPar
\sphinxstylestrong{Lv 1. A walkthrough with a group:}

\sphinxAtStartPar
\sphinxstyleemphasis{Preparation and time:} 1 Day for the publisher, 20 min. for the group

\sphinxAtStartPar
The publisher: understand the basic of how to stream geometry with Radii Grashopper and the basics on operating the viewer.
The observers: Install the viewer, log into the server and navigate

\sphinxAtStartPar
\sphinxstyleemphasis{Tech:}

\sphinxAtStartPar
The Designer: min. a laptop of medium strength for streaming and running the viewer in parallel
The observers:  min. a laptop of medium strength for running the viewer

\sphinxAtStartPar
\sphinxstylestrong{Lv 2. Class of students work and develop their projects through a semester with Radii:}

\sphinxAtStartPar
\sphinxstyleemphasis{Preparation and time:} 3\sphinxhyphen{}4 days of preparation to understand the program and be able to teach it.

\sphinxAtStartPar
We teach Radii in 3 seatings throughout the semester in a total of 5 hours, with constant usage in between.
This enables the student to test models in Radii as well as present them at the end.

\sphinxAtStartPar
The first 1,5 hours explain the basics on how to use the viewer and the grashopper plugin
For the viewer this includes how to connect to a channel and navigate in a model. In grashopper
is is about the most basic workflow and most important publishing components.
In the second sitting we introduce the saving to the cloud and locally in radii files, how to animate, define views and basics of publish control.
The third class goes into detail about the scenario manager that is part of publish control and leaves time for possible subscribe components.

\sphinxAtStartPar
\sphinxstyleemphasis{Tech:}
\begin{itemize}
\item {} 
\sphinxAtStartPar
Personal laptops of medium strength for each student

\item {} 
\sphinxAtStartPar
Min. one strong PC for presentations for the viewer to run on
\begin{itemize}
\item {} 
\sphinxAtStartPar
ideally one per group to allow for stronger access and testing

\end{itemize}

\item {} 
\sphinxAtStartPar
A bigger screen/projector relative to the size of the class

\item {} 
\sphinxAtStartPar
Optional:
\begin{itemize}
\item {} 
\sphinxAtStartPar
Oculus or other Vr devices (at the time of writing only Oculus is supported)

\item {} 
\sphinxAtStartPar
Phones or Ipads for augmented reality usage

\item {} 
\sphinxAtStartPar
Open space to walk around in VR/Ar

\end{itemize}

\end{itemize}

\sphinxAtStartPar
\sphinxstylestrong{Lv 3. Interactive collaborative work}

\sphinxAtStartPar
\sphinxstylestrong{TO BE ADDED}

\sphinxstepscope


\section{Radii Viewer Overview}
\label{\detokenize{tutorial/Viewer_PC/documentation_rst/0_Viewer:radii-viewer-overview}}\label{\detokenize{tutorial/Viewer_PC/documentation_rst/0_Viewer::doc}}
\sphinxAtStartPar
The RADii Viewer is a program used to connect to the RADii channels and explore the models.
It can be run on Win/ Mac, Oculus, iOS 11+ Android 7+ and a browser.

\sphinxAtStartPar
In the following chapters you can find a detailed documentation of all parts of the Radii Viewer. Please keep in mind that this project is still in
development and might have changed over time.

\sphinxAtStartPar
In case you might find something unclear, out of date or missing please contact the Chair of Gramazio \& Kohler at ETHZ \sphinxhref{mailto:gesievi@ethz.ch}{gesievi@ethz.ch}

\sphinxAtStartPar
\sphinxstylestrong{Keyboard Assignments}


\begin{savenotes}\sphinxattablestart
\sphinxthistablewithglobalstyle
\centering
\begin{tabulary}{\linewidth}[t]{TT}
\sphinxtoprule
\sphinxtableatstartofbodyhook
\sphinxAtStartPar
Movement
&
\sphinxAtStartPar
W,A,S,D
\\
\sphinxhline
\sphinxAtStartPar
Jump
&
\sphinxAtStartPar
Space
\\
\sphinxhline
\sphinxAtStartPar
Shift
&
\sphinxAtStartPar
Sprint
\\
\sphinxhline
\sphinxAtStartPar
Q Fly
&
\sphinxAtStartPar
Up
\\
\sphinxhline
\sphinxAtStartPar
E Fly
&
\sphinxAtStartPar
Down
\\
\sphinxhline
\sphinxAtStartPar
Right Mouse
&
\sphinxAtStartPar
Main Menu (open,close)(3.1\sphinxhyphen{}3.5)
\\
\sphinxhline
\sphinxAtStartPar
Left Mouse
&
\sphinxAtStartPar
Operation Menu (3.6)
\\
\sphinxbottomrule
\end{tabulary}
\sphinxtableafterendhook\par
\sphinxattableend\end{savenotes}

\sphinxAtStartPar
\sphinxstylestrong{Viewer Page Menu:}

\sphinxstepscope


\subsection{Connect Menu}
\label{\detokenize{tutorial/Viewer_PC/documentation_rst/1_Connect:connect-menu}}\label{\detokenize{tutorial/Viewer_PC/documentation_rst/1_Connect::doc}}
\noindent\sphinxincludegraphics[scale=2.0]{{connect}.png}

\noindent\sphinxincludegraphics{{Menu_connect}.png}
\begin{enumerate}
\sphinxsetlistlabels{\arabic}{enumi}{enumii}{}{.}%
\item {} 
\sphinxAtStartPar
Custom Account from Radii.info
\begin{itemize}
\item {} 
\sphinxAtStartPar
Necessary if you want to edit items in cooperation with others

\end{itemize}

\item {} 
\sphinxAtStartPar
Account
\begin{itemize}
\item {} 
\sphinxAtStartPar
Lists all channels you have access to and your rights (owner/editor/administrator)

\end{itemize}

\item {} 
\sphinxAtStartPar
Connection
\begin{itemize}
\item {} 
\sphinxAtStartPar
Channel \sphinxhyphen{} input the channel address and the subchannel you want to access, separated by a point: channel name.subchannelname

\item {} 
\sphinxAtStartPar
Nickname \sphinxhyphen{} displayed for others on the server

\end{itemize}

\item {} 
\sphinxAtStartPar
Subscription
\begin{itemize}
\item {} 
\sphinxAtStartPar
The types of data you receive: geometry, lines, point\sphinxhyphen{}clouds, textures, other players, views, messages, settings

\end{itemize}

\item {} 
\sphinxAtStartPar
Server
\begin{itemize}
\item {} 
\sphinxAtStartPar
I \sphinxhyphen{} Info menu with version number, radii.info, Privacy Policy and Terms and Conditions

\item {} 
\sphinxAtStartPar
Press Connect after setting the channel name

\item {} 
\sphinxAtStartPar
Disconnect to leave the channel

\item {} \begin{description}
\sphinxlineitem{Sync \sphinxhyphen{} synchronize}\begin{itemize}
\item {} 
\sphinxAtStartPar
you will receive all data that is being sent to the channel at the moment

\end{itemize}

\end{description}

\end{itemize}

\end{enumerate}

\sphinxstepscope


\subsection{Remote Content Menu}
\label{\detokenize{tutorial/Viewer_PC/documentation_rst/2_Remote_content:remote-content-menu}}\label{\detokenize{tutorial/Viewer_PC/documentation_rst/2_Remote_content::doc}}
\noindent\sphinxincludegraphics{{Menu_remote_content}.png}
\begin{enumerate}
\sphinxsetlistlabels{\arabic}{enumi}{enumii}{}{.}%
\setcounter{enumi}{5}
\item {} 
\sphinxAtStartPar
This menu displays data stored on the channel, once you input the channel address.

\item {} \begin{description}
\sphinxlineitem{Opens the save menu}\begin{itemize}
\item {} 
\sphinxAtStartPar
Export = exports a radii locally

\item {} 
\sphinxAtStartPar
Import = imports a radii file from a path

\item {} 
\sphinxAtStartPar
Save = saves a radii file locally in a standard path

\end{itemize}

\end{description}

\end{enumerate}

\sphinxstepscope


\subsection{Publish Menu}
\label{\detokenize{tutorial/Viewer_PC/documentation_rst/3_Publish:publish-menu}}\label{\detokenize{tutorial/Viewer_PC/documentation_rst/3_Publish::doc}}
\noindent\sphinxincludegraphics{{Menu_publish}.png}
\begin{enumerate}
\sphinxsetlistlabels{\arabic}{enumi}{enumii}{}{.}%
\setcounter{enumi}{7}
\item {} 
\sphinxAtStartPar
Parameters \sphinxhyphen{} send parameters back to grashopper
\begin{itemize}
\item {} \begin{description}
\sphinxlineitem{Add by typing a name and click:}\begin{itemize}
\item {} 
\sphinxAtStartPar
Boolean (on/off)

\item {} 
\sphinxAtStartPar
Slider (Number between 0 and 1)

\end{itemize}

\end{description}

\item {} 
\sphinxAtStartPar
To receive them in grasshopper use the SubscribeParameter component

\end{itemize}

\end{enumerate}

\sphinxstepscope


\subsection{Users Menu}
\label{\detokenize{tutorial/Viewer_PC/documentation_rst/4_Users:users-menu}}\label{\detokenize{tutorial/Viewer_PC/documentation_rst/4_Users::doc}}
\noindent\sphinxincludegraphics{{Menu_users}.png}

\sphinxAtStartPar
\sphinxstylestrong{Users \sphinxhyphen{} List of all data in the viewer}
\begin{enumerate}
\sphinxsetlistlabels{\arabic}{enumi}{enumii}{}{.}%
\setcounter{enumi}{8}
\item {} 
\sphinxAtStartPar
Import / Editor / Viewer
\begin{itemize}
\item {} 
\sphinxAtStartPar
Import = data loaded from RADii file
\begin{itemize}
\item {} 
\sphinxAtStartPar
Content \sphinxhyphen{} toggle content

\item {} 
\sphinxAtStartPar
Clear \sphinxhyphen{} delete content from viewer

\end{itemize}

\item {} 
\sphinxAtStartPar
Editor = data that was sent from RhinoGrashopper
\begin{itemize}
\item {} 
\sphinxAtStartPar
Follow \sphinxhyphen{} follow the rhino view of the user

\item {} 
\sphinxAtStartPar
User toggle avatar

\item {} 
\sphinxAtStartPar
Content \sphinxhyphen{} toggle content

\item {} 
\sphinxAtStartPar
Clear \sphinxhyphen{} delete content from viewer

\end{itemize}

\item {} 
\sphinxAtStartPar
Viewer  all other viewers that are on the channel

\end{itemize}

\item {} 
\sphinxAtStartPar
Data \& Cameras = toggle labels
\begin{itemize}
\item {} 
\sphinxAtStartPar
Clear = delete all content from viewer

\end{itemize}

\end{enumerate}

\sphinxstepscope


\subsection{World Menu}
\label{\detokenize{tutorial/Viewer_PC/documentation_rst/5_World:world-menu}}\label{\detokenize{tutorial/Viewer_PC/documentation_rst/5_World::doc}}
\noindent\sphinxincludegraphics{{Menu_world}.png}
\begin{enumerate}
\sphinxsetlistlabels{\arabic}{enumi}{enumii}{}{.}%
\setcounter{enumi}{10}
\item {} 
\sphinxAtStartPar
Camera
\begin{itemize}
\item {} 
\sphinxAtStartPar
Field of view \sphinxhyphen{} controls the field of view at the cost of some distortion

\item {} 
\sphinxAtStartPar
Sensitivity \sphinxhyphen{} mouse sensitivity

\item {} 
\sphinxAtStartPar
Speed \sphinxhyphen{} movement speed

\end{itemize}

\end{enumerate}
\begin{quote}

\sphinxAtStartPar
\sphinxstylestrong{Note:} turn slow when someone is following you through the project
\end{quote}
\begin{enumerate}
\sphinxsetlistlabels{\arabic}{enumi}{enumii}{}{.}%
\setcounter{enumi}{11}
\item {} 
\sphinxAtStartPar
Effects \sphinxhyphen{} turning them off increases performance
\begin{itemize}
\item {} 
\sphinxAtStartPar
Motion = motion blur

\item {} 
\sphinxAtStartPar
Bloom = makes bright spots bleed at the edges, simulating a real camera

\item {} 
\sphinxAtStartPar
DOF = depth of field \sphinxhyphen{} distance between closest and furthest part of an image that are in focus

\item {} 
\sphinxAtStartPar
Chrom = chromatic effect \sphinxhyphen{} adds artifacts to the image, simulating a poor len

\item {} 
\sphinxAtStartPar
Vignet = darkening on the edges of images, simulating real cameras

\item {} 
\sphinxAtStartPar
Inverse = clipping/sectioning leaves a ghost of the hidden geometry

\end{itemize}

\item {} 
\sphinxAtStartPar
Point Cloud
\begin{itemize}
\item {} 
\sphinxAtStartPar
Point Size

\item {} 
\sphinxAtStartPar
Point near size = increases point sizes near you

\end{itemize}

\end{enumerate}
\begin{quote}

\sphinxAtStartPar
\sphinxstylestrong{Note:} point clouds are disabled in IOS/Android viewers because they require a lot of computing power
\end{quote}
\begin{enumerate}
\sphinxsetlistlabels{\arabic}{enumi}{enumii}{}{.}%
\setcounter{enumi}{13}
\item {} 
\sphinxAtStartPar
Weather
\begin{itemize}
\item {} 
\sphinxAtStartPar
Quality = resolution of the sky, above lv3 not significantly better

\item {} 
\sphinxAtStartPar
Condition = types of weather: rainy, foggy and sunny

\item {} 
\sphinxAtStartPar
Fog density = can hide the horizon but also your model

\end{itemize}

\end{enumerate}
\begin{quote}

\sphinxAtStartPar
\sphinxstylestrong{Note:} for better performance: turn weather to sunny, fog off, quality to lowest
\end{quote}
\begin{enumerate}
\sphinxsetlistlabels{\arabic}{enumi}{enumii}{}{.}%
\setcounter{enumi}{14}
\item {} 
\sphinxAtStartPar
Time

\item {} 
\sphinxAtStartPar
Transform
\begin{itemize}
\item {} 
\sphinxAtStartPar
Rotation = rotates all models around the 0 point

\item {} 
\sphinxAtStartPar
Scale

\end{itemize}

\item {} 
\sphinxAtStartPar
Select Viewer
\begin{itemize}
\item {} \begin{description}
\sphinxlineitem{Standard}\begin{itemize}
\item {} 
\sphinxAtStartPar
Stereo Shutter = for active 3D glasses

\item {} 
\sphinxAtStartPar
Chroma Key = a virtual green screen

\item {} 
\sphinxAtStartPar
Pepper Ghost = displays the model in a virtual box

\item {} 
\sphinxAtStartPar
Augmented Reality (AR) = displays virtual model in a real environment \sphinxhyphen{} available on IOs, Android, Oculus

\end{itemize}

\end{description}

\item {} 
\sphinxAtStartPar
Tracker Method = setting for AR devices

\item {} \begin{description}
\sphinxlineitem{Projection = screen size settings}\begin{itemize}
\item {} 
\sphinxAtStartPar
you can also edit the overall scale, height of horizon

\end{itemize}

\end{description}

\item {} 
\sphinxAtStartPar
Grid = toggles the default floor

\item {} 
\sphinxAtStartPar
Origin = toggles the origin point

\end{itemize}

\end{enumerate}

\sphinxstepscope


\subsection{Operation Menu}
\label{\detokenize{tutorial/Viewer_PC/documentation_rst/6_Operation_menu:operation-menu}}\label{\detokenize{tutorial/Viewer_PC/documentation_rst/6_Operation_menu::doc}}
\noindent\sphinxincludegraphics{{Menu_action_detail_number}.png}

\sphinxAtStartPar
Open the Operation menu with a left click, close it with a right click


\begin{savenotes}\sphinxattablestart
\sphinxthistablewithglobalstyle
\centering
\begin{tabulary}{\linewidth}[t]{TTT}
\sphinxtoprule
\sphinxtableatstartofbodyhook
\sphinxAtStartPar
\sphinxstylestrong{Home}  = moves you to the origin
&
\sphinxAtStartPar
\sphinxstylestrong{Fly} = makes you fly and turns off collision for you
&
\sphinxAtStartPar
\sphinxstylestrong{Delete} = delete selected object
\\
\sphinxhline
\sphinxAtStartPar
\sphinxstylestrong{Cap}   = screen capture
&
\sphinxAtStartPar
\sphinxstylestrong{Go To} = move to a point
&
\sphinxAtStartPar
\sphinxstylestrong{Point} = (laser) pointer visible to all in the same channel
\\
\sphinxhline
\sphinxAtStartPar
\sphinxstylestrong{Grab}  = move a part of a model \sphinxhyphen{} changes made are visible to others if geometry is published as shared to \sphinxstyleemphasis{copy:} point on geometry hold Ctrl + Left mouse
&
\sphinxAtStartPar
\sphinxstylestrong{X} = left click to close the menu or right click anywhere
&
\sphinxAtStartPar
\sphinxstylestrong{Scale} = scale a selected model
\\
\sphinxhline
\sphinxAtStartPar
\sphinxstylestrong{Set}   = set the section
&
\sphinxAtStartPar
\sphinxstylestrong{Move} = move the section
&
\sphinxAtStartPar
\sphinxstylestrong{Clip}  = toggles sections \sphinxhyphen{} to display the geometry shadow toggle World menu \sphinxhyphen{}\textgreater{} Effects \sphinxhyphen{}\textgreater{} Inverse
\\
\sphinxbottomrule
\end{tabulary}
\sphinxtableafterendhook\par
\sphinxattableend\end{savenotes}

\sphinxstepscope


\section{Grashopper Overview}
\label{\detokenize{tutorial/grashopper/documentation_rst/01_Components_Overview:grashopper-overview}}\label{\detokenize{tutorial/grashopper/documentation_rst/01_Components_Overview::doc}}
\noindent\sphinxincludegraphics[scale=1.2]{{all_components}.png}

\sphinxAtStartPar
There are 6 types of components:

\begin{DUlineblock}{0em}
\item[] \sphinxstylestrong{Connect} is the fundamental component to connect to a channel, it is always connected to all components that are in use.
\item[] \sphinxstylestrong{Params} relay the described datatype
\item[] \sphinxstylestrong{Publishing} components can send data to a channel and its connected receivers.
\item[] \sphinxstylestrong{Save} enables you to save locally to .radii files or on a channel in the cloud.
\item[] \sphinxstylestrong{Subscribe} imports data from from a connected channel to your local Rhino Grashopper session. The data can then be further processed or saved.
\item[] \sphinxstylestrong{Tools} to modify point clouds
\end{DUlineblock}


\subsection{Grashopper Components}
\label{\detokenize{tutorial/grashopper/documentation_rst/01_Components_Overview:grashopper-components}}
\sphinxAtStartPar
\sphinxstylestrong{Publish Components}

\sphinxstepscope


\subsubsection{Connect}
\label{\detokenize{tutorial/grashopper/documentation_rst/02_connect:connect}}\label{\detokenize{tutorial/grashopper/documentation_rst/02_connect::doc}}
\sphinxAtStartPar
The connect component is the central component of the radii plugin, every other component is connected to the connection part.

\sphinxAtStartPar
\sphinxstylestrong{Input}


\begin{savenotes}\sphinxattablestart
\sphinxthistablewithglobalstyle
\centering
\begin{tabulary}{\linewidth}[t]{TTT}
\sphinxtoprule
\sphinxstyletheadfamily 
\sphinxAtStartPar
Name
&\sphinxstyletheadfamily 
\sphinxAtStartPar
Description
&\sphinxstyletheadfamily 
\sphinxAtStartPar
Type
\\
\sphinxmidrule
\sphinxtableatstartofbodyhook
\sphinxAtStartPar
Connect
&
\sphinxAtStartPar
Start the connection to the server
&
\sphinxAtStartPar
Boolean
\\
\sphinxhline
\sphinxAtStartPar
Point to
&
\sphinxAtStartPar
The rhino view is sending a pointer
&
\sphinxAtStartPar
Boolean
\\
\sphinxhline
\sphinxAtStartPar
Follow
&
\sphinxAtStartPar
Everyone follows the rhino view
&
\sphinxAtStartPar
Boolean
\\
\sphinxbottomrule
\end{tabulary}
\sphinxtableafterendhook\par
\sphinxattableend\end{savenotes}

\sphinxAtStartPar
\sphinxstylestrong{Output}


\begin{savenotes}\sphinxattablestart
\sphinxthistablewithglobalstyle
\centering
\begin{tabulary}{\linewidth}[t]{TTT}
\sphinxtoprule
\sphinxstyletheadfamily 
\sphinxAtStartPar
Name
&\sphinxstyletheadfamily 
\sphinxAtStartPar
Description
&\sphinxstyletheadfamily 
\sphinxAtStartPar
Type
\\
\sphinxmidrule
\sphinxtableatstartofbodyhook
\sphinxAtStartPar
Connection
&
\sphinxAtStartPar
All further components have to be connected here
&
\sphinxAtStartPar
Radii components
\\
\sphinxbottomrule
\end{tabulary}
\sphinxtableafterendhook\par
\sphinxattableend\end{savenotes}

\noindent\sphinxincludegraphics[scale=0.8]{{Connect_gh}.png}

\sphinxAtStartPar
Right click menu:

\noindent\sphinxincludegraphics{{Connect.1}.png}
\begin{itemize}
\item {} 
\sphinxAtStartPar
Give yourself a nickname \sphinxhyphen{} this will be your visible name in the viewer

\item {} \begin{description}
\sphinxlineitem{input your:}\begin{itemize}
\item {} 
\sphinxAtStartPar
User Name

\item {} 
\sphinxAtStartPar
Password

\end{itemize}

\end{description}

\item {} 
\sphinxAtStartPar
click on load account

\item {} 
\sphinxAtStartPar
if you do not have an account create one on radii.info

\end{itemize}

\sphinxAtStartPar
Note: please use your ETH\sphinxhyphen{}mail so we can connect you to the studio

\noindent\sphinxincludegraphics{{Connect.11}.png}

\sphinxstepscope


\subsubsection{PublishGeometry}
\label{\detokenize{tutorial/grashopper/documentation_rst/03_publish_geometry:publishgeometry}}\label{\detokenize{tutorial/grashopper/documentation_rst/03_publish_geometry::doc}}
\noindent\sphinxincludegraphics[scale=0.8]{{Publish_geometry}.png}

\sphinxAtStartPar
\sphinxstylestrong{Input}


\begin{savenotes}\sphinxattablestart
\sphinxthistablewithglobalstyle
\centering
\begin{tabulary}{\linewidth}[t]{TTT}
\sphinxtoprule
\sphinxstyletheadfamily 
\sphinxAtStartPar
Name
&\sphinxstyletheadfamily 
\sphinxAtStartPar
Description
&\sphinxstyletheadfamily 
\sphinxAtStartPar
Type
\\
\sphinxmidrule
\sphinxtableatstartofbodyhook
\sphinxAtStartPar
Connection
&
\sphinxAtStartPar
Link with the Connect component
&
\sphinxAtStartPar
Connection
\\
\sphinxhline
\sphinxAtStartPar
Geometry
&
\sphinxAtStartPar
Geometry you want to upload
&
\sphinxAtStartPar
Brep or Mesh
\\
\sphinxhline
\sphinxAtStartPar
Quality
&
\sphinxAtStartPar
Mesh Quality for conversion
&
\sphinxAtStartPar
Settings
\\
\sphinxbottomrule
\end{tabulary}
\sphinxtableafterendhook\par
\sphinxattableend\end{savenotes}

\sphinxAtStartPar
\sphinxstylestrong{Quality:}
To control the quality of your geometry you can use the following components
\begin{itemize}
\item {} 
\sphinxAtStartPar
Setting (Speed)

\item {} 
\sphinxAtStartPar
Setting (Quality)

\item {} 
\sphinxAtStartPar
Setting (Custom)
\begin{itemize}
\item {} 
\sphinxAtStartPar
use “Min Edge” to sett the minimal edge length this will make your model low poly if you go to high

\end{itemize}

\end{itemize}

\sphinxAtStartPar
\sphinxstylestrong{Output}


\begin{savenotes}\sphinxattablestart
\sphinxthistablewithglobalstyle
\centering
\begin{tabulary}{\linewidth}[t]{TTT}
\sphinxtoprule
\sphinxstyletheadfamily 
\sphinxAtStartPar
Name
&\sphinxstyletheadfamily 
\sphinxAtStartPar
Description
&\sphinxstyletheadfamily 
\sphinxAtStartPar
Type
\\
\sphinxmidrule
\sphinxtableatstartofbodyhook
\sphinxAtStartPar
Log
&
\sphinxAtStartPar
Documents changes \& data send
&
\sphinxAtStartPar
Text
\\
\sphinxhline
\sphinxAtStartPar
Save
&
\sphinxAtStartPar
Connect to Save component for saving
&
\sphinxAtStartPar
Radii content
\\
\sphinxbottomrule
\end{tabulary}
\sphinxtableafterendhook\par
\sphinxattableend\end{savenotes}
\begin{itemize}
\item {} 
\sphinxAtStartPar
Log can help identify how much data and what kind of it is send

\end{itemize}

\sphinxAtStartPar
\sphinxstylestrong{Menu:}


\begin{savenotes}\sphinxattablestart
\sphinxthistablewithglobalstyle
\centering
\begin{tabulary}{\linewidth}[t]{TT}
\sphinxtoprule
\sphinxtableatstartofbodyhook
\sphinxAtStartPar
Update:
&
\sphinxAtStartPar
update only changed geometry
\\
\sphinxhline
\sphinxAtStartPar
Rebuild:
&
\sphinxAtStartPar
republish everything in the component
\\
\sphinxhline
\sphinxAtStartPar
Render:
&
\sphinxAtStartPar
visible/invisible
\\
\sphinxhline
\sphinxAtStartPar
Collision:
&
\sphinxAtStartPar
permeable/impermeable
\\
\sphinxhline
\sphinxAtStartPar
Physics:
&
\sphinxAtStartPar
objects push on each other
\\
\sphinxhline
\sphinxAtStartPar
Gravity:
&
\sphinxAtStartPar
9.807 m/s\(\sp{\text{2}}\) pulling on each object
\\
\sphinxhline
\sphinxAtStartPar
Shared:
&
\sphinxAtStartPar
collaborative editing of geometry in the viewer
\\
\sphinxbottomrule
\end{tabulary}
\sphinxtableafterendhook\par
\sphinxattableend\end{savenotes}

\sphinxstepscope


\subsubsection{PublishMaterial}
\label{\detokenize{tutorial/grashopper/documentation_rst/04_publish_material:publishmaterial}}\label{\detokenize{tutorial/grashopper/documentation_rst/04_publish_material::doc}}
\noindent\sphinxincludegraphics{{Publish_Textures}.png}

\sphinxAtStartPar
\sphinxstylestrong{Input}


\begin{savenotes}\sphinxattablestart
\sphinxthistablewithglobalstyle
\centering
\begin{tabulary}{\linewidth}[t]{TTT}
\sphinxtoprule
\sphinxstyletheadfamily 
\sphinxAtStartPar
Name
&\sphinxstyletheadfamily 
\sphinxAtStartPar
Description
&\sphinxstyletheadfamily 
\sphinxAtStartPar
Type
\\
\sphinxmidrule
\sphinxtableatstartofbodyhook
\sphinxAtStartPar
Connection
&
\sphinxAtStartPar
Link with the Connect component
&
\sphinxAtStartPar
Connection
\\
\sphinxbottomrule
\end{tabulary}
\sphinxtableafterendhook\par
\sphinxattableend\end{savenotes}

\sphinxAtStartPar
\sphinxstylestrong{Output}


\begin{savenotes}\sphinxattablestart
\sphinxthistablewithglobalstyle
\centering
\begin{tabulary}{\linewidth}[t]{TTT}
\sphinxtoprule
\sphinxstyletheadfamily 
\sphinxAtStartPar
Name
&\sphinxstyletheadfamily 
\sphinxAtStartPar
Description
&\sphinxstyletheadfamily 
\sphinxAtStartPar
Type
\\
\sphinxmidrule
\sphinxtableatstartofbodyhook
\sphinxAtStartPar
Log
&
\sphinxAtStartPar
Documents changes \& Data send
&
\sphinxAtStartPar
Text
\\
\sphinxhline
\sphinxAtStartPar
Save
&
\sphinxAtStartPar
Connect to SaveContent for saving
&
\sphinxAtStartPar
Radii content
\\
\sphinxbottomrule
\end{tabulary}
\sphinxtableafterendhook\par
\sphinxattableend\end{savenotes}

\sphinxstepscope


\subsubsection{PublishSection}
\label{\detokenize{tutorial/grashopper/documentation_rst/05_publish_section:publishsection}}\label{\detokenize{tutorial/grashopper/documentation_rst/05_publish_section::doc}}
\noindent\sphinxincludegraphics[scale=0.8]{{Publish_section}.png}

\sphinxAtStartPar
\sphinxstylestrong{Input}


\begin{savenotes}\sphinxattablestart
\sphinxthistablewithglobalstyle
\centering
\begin{tabulary}{\linewidth}[t]{TTT}
\sphinxtoprule
\sphinxstyletheadfamily 
\sphinxAtStartPar
Name
&\sphinxstyletheadfamily 
\sphinxAtStartPar
Description
&\sphinxstyletheadfamily 
\sphinxAtStartPar
Type
\\
\sphinxmidrule
\sphinxtableatstartofbodyhook
\sphinxAtStartPar
Connection
&
\sphinxAtStartPar
Link with the Connect component
&
\sphinxAtStartPar
Connection
\\
\sphinxhline
\sphinxAtStartPar
Section
&
\sphinxAtStartPar
Plane A plane that will cut the model
&
\sphinxAtStartPar
Plane/surfaces
\\
\sphinxhline
\sphinxAtStartPar
Index
&
\sphinxAtStartPar
Select a plane from a list
&
\sphinxAtStartPar
Number
\\
\sphinxbottomrule
\end{tabulary}
\sphinxtableafterendhook\par
\sphinxattableend\end{savenotes}
\begin{itemize}
\item {} 
\sphinxAtStartPar
at the time of writing index only works for a list of surfaces in the input: section plane

\item {} 
\sphinxAtStartPar
Rhino clipping planes have to be selected in the menu at \sphinxhyphen{}Activate clipping planes\sphinxhyphen{}

\end{itemize}

\sphinxAtStartPar
\sphinxstylestrong{Output}


\begin{savenotes}\sphinxattablestart
\sphinxthistablewithglobalstyle
\centering
\begin{tabulary}{\linewidth}[t]{TTT}
\sphinxtoprule
\sphinxstyletheadfamily 
\sphinxAtStartPar
Name
&\sphinxstyletheadfamily 
\sphinxAtStartPar
Description
&\sphinxstyletheadfamily 
\sphinxAtStartPar
Type
\\
\sphinxmidrule
\sphinxtableatstartofbodyhook
\sphinxAtStartPar
Log
&
\sphinxAtStartPar
Documents changes \& Data send
&
\sphinxAtStartPar
Text
\\
\sphinxhline
\sphinxAtStartPar
Save
&
\sphinxAtStartPar
Connect to SaveContent for saving
&
\sphinxAtStartPar
Radii content
\\
\sphinxbottomrule
\end{tabulary}
\sphinxtableafterendhook\par
\sphinxattableend\end{savenotes}

\sphinxAtStartPar
\sphinxstylestrong{Menu}


\begin{savenotes}\sphinxattablestart
\sphinxthistablewithglobalstyle
\centering
\begin{tabulary}{\linewidth}[t]{TT}
\sphinxtoprule
\sphinxtableatstartofbodyhook
\sphinxAtStartPar
Section Duration:
&
\sphinxAtStartPar
how quickly the section is moving into place
\\
\sphinxhline
\sphinxAtStartPar
Set duration:
&
\sphinxAtStartPar
toggle to have a moving clipping plane
\\
\sphinxhline
\sphinxAtStartPar
Activate clipping planes:
&
\sphinxAtStartPar
clipping planes from rhino to be selected
\\
\sphinxbottomrule
\end{tabulary}
\sphinxtableafterendhook\par
\sphinxattableend\end{savenotes}

\sphinxAtStartPar
\sphinxstylestrong{Note:}
\begin{itemize}
\item {} 
\sphinxAtStartPar
In Rhino flipping a clipping plane is not recognized by Radii, rotating the plane by 180° however archives the same

\item {} 
\sphinxAtStartPar
Names of clipping planes are not carried over into the grashopper plugin

\end{itemize}

\sphinxstepscope


\subsubsection{PublishControl}
\label{\detokenize{tutorial/grashopper/documentation_rst/06_publish_control:publishcontrol}}\label{\detokenize{tutorial/grashopper/documentation_rst/06_publish_control::doc}}
\noindent\sphinxincludegraphics[scale=0.8]{{Publish__controll}.png}
\begin{description}
\sphinxlineitem{Publish control lets you control the viewer settings from grashopper}\begin{itemize}
\item {} 
\sphinxAtStartPar
Everything you set as input can be saved in this publish control component with the scenario manager. Be careful with geometry content, this can make your grasshopper file very heavy.

\sphinxAtStartPar
Note: Grasshopper has an autosave. If the Publish Control component becomes too heavy, it will make you wait a lot

\end{itemize}

\end{description}

\sphinxAtStartPar
\sphinxstylestrong{Input}


\begin{savenotes}\sphinxattablestart
\sphinxthistablewithglobalstyle
\centering
\begin{tabulary}{\linewidth}[t]{TTT}
\sphinxtoprule
\sphinxstyletheadfamily 
\sphinxAtStartPar
Name
&\sphinxstyletheadfamily 
\sphinxAtStartPar
Description
&\sphinxstyletheadfamily 
\sphinxAtStartPar
Type
\\
\sphinxmidrule
\sphinxtableatstartofbodyhook
\sphinxAtStartPar
Connection
&
\sphinxAtStartPar
Link with the Connect component
&
\sphinxAtStartPar
Connect
\\
\sphinxhline
\sphinxAtStartPar
Time of Year
&
\sphinxAtStartPar
Day of the Year
&
\sphinxAtStartPar
Number
\\
\sphinxhline
\sphinxAtStartPar
Time of Day
&
\sphinxAtStartPar
Time of the day
&
\sphinxAtStartPar
Number
\\
\sphinxhline
\sphinxAtStartPar
Content Save
&
\sphinxAtStartPar
output from other radii components
&
\sphinxAtStartPar
save (Radii)
\\
\sphinxhline
\sphinxAtStartPar
Index
&
\sphinxAtStartPar
For switching between scenarios
&
\sphinxAtStartPar
Number
\\
\sphinxbottomrule
\end{tabulary}
\sphinxtableafterendhook\par
\sphinxattableend\end{savenotes}

\sphinxAtStartPar
\sphinxstylestrong{Output}


\begin{savenotes}\sphinxattablestart
\sphinxthistablewithglobalstyle
\centering
\begin{tabulary}{\linewidth}[t]{TTT}
\sphinxtoprule
\sphinxstyletheadfamily 
\sphinxAtStartPar
Name
&\sphinxstyletheadfamily 
\sphinxAtStartPar
Description
&\sphinxstyletheadfamily 
\sphinxAtStartPar
Type
\\
\sphinxmidrule
\sphinxtableatstartofbodyhook
\sphinxAtStartPar
Log
&
\sphinxAtStartPar
Documents changes \& Data send
&
\sphinxAtStartPar
Text
\\
\sphinxhline
\sphinxAtStartPar
Save Control
&
\sphinxAtStartPar
Saving the controls only
&
\sphinxAtStartPar
Radii
\\
\sphinxhline
\sphinxAtStartPar
Save Scenario
&
\sphinxAtStartPar
Save control and geometry
&
\sphinxAtStartPar
Radii
\\
\sphinxbottomrule
\end{tabulary}
\sphinxtableafterendhook\par
\sphinxattableend\end{savenotes}

\sphinxAtStartPar
\sphinxstylestrong{Menu}


\begin{savenotes}\sphinxattablestart
\sphinxthistablewithglobalstyle
\centering
\begin{tabulary}{\linewidth}[t]{TT}
\sphinxtoprule
\sphinxstyletheadfamily 
\sphinxAtStartPar
Name
&\sphinxstyletheadfamily 
\sphinxAtStartPar
Description
\\
\sphinxmidrule
\sphinxtableatstartofbodyhook
\sphinxAtStartPar
Scale
&
\sphinxAtStartPar
Set the model scale
\\
\sphinxhline
\sphinxAtStartPar
Weather
&
\sphinxAtStartPar
Weather options
\\
\sphinxhline
\sphinxAtStartPar
Grid
&
\sphinxAtStartPar
Toggle base floor
\\
\sphinxhline
\sphinxAtStartPar
Origin
&
\sphinxAtStartPar
Toggle origin sign
\\
\sphinxhline
\sphinxAtStartPar
Labels
&
\sphinxAtStartPar
Toggle all labels
\\
\sphinxhline
\sphinxAtStartPar
Fly
&
\sphinxAtStartPar
Forces viewers to fly
\\
\sphinxhline
\sphinxAtStartPar
Set position
&
\sphinxAtStartPar
Set the camera of your active rhino viewport as position
\\
\sphinxhline
\sphinxAtStartPar
Save position
&
\sphinxAtStartPar
Include the position in a scenario save
\\
\sphinxhline
\sphinxAtStartPar
Clear
&
\sphinxAtStartPar
Clears all content from viewers
\\
\sphinxhline
\sphinxAtStartPar
Set Location
&
\sphinxAtStartPar
Sets the world location for the sun
\\
\sphinxhline
\sphinxAtStartPar
Origin rotation
&
\sphinxAtStartPar
Rotates the model by x\sphinxhyphen{}degrees
\\
\sphinxhline
\sphinxAtStartPar
Set origin rotation
&
\sphinxAtStartPar
Confirm rotation
\\
\sphinxhline
\sphinxAtStartPar
File
&
\sphinxAtStartPar
Displays the files from the channel that you can input below
\\
\sphinxhline
\sphinxAtStartPar
Get files
&
\sphinxAtStartPar
Connects to the channel, now the saved files appear in files
\\
\sphinxhline
\sphinxAtStartPar
Scenario Manager
&
\sphinxAtStartPar
Opens the scenario manager, you can save your scenario with all settings from above
\\
\sphinxbottomrule
\end{tabulary}
\sphinxtableafterendhook\par
\sphinxattableend\end{savenotes}
\begin{itemize}
\item {} 
\sphinxAtStartPar
with files you can command the scenario manager to directly download a file from the server instead of uploading it and sending it to the viewers

\end{itemize}


\paragraph{Scenario Manager}
\label{\detokenize{tutorial/grashopper/documentation_rst/06_publish_control:scenario-manager}}
\noindent\sphinxincludegraphics[scale=0.8]{{Publish__controll_manager}.png}

\sphinxAtStartPar
The scenario manager lets you save the selected options of the publishControl component and content that is connected to it.


\begin{savenotes}\sphinxattablestart
\sphinxthistablewithglobalstyle
\centering
\begin{tabulary}{\linewidth}[t]{TT}
\sphinxtoprule
\sphinxtableatstartofbodyhook
\sphinxAtStartPar
Clear
&
\sphinxAtStartPar
clears the scene between each scenario \sphinxhyphen{} this can mean that you are uploading your geometry every time your load a new one
\\
\sphinxhline
\sphinxAtStartPar
Duration
&
\sphinxAtStartPar
length of the scenario
\\
\sphinxbottomrule
\end{tabulary}
\sphinxtableafterendhook\par
\sphinxattableend\end{savenotes}

\sphinxAtStartPar
\sphinxstylestrong{Column descriptions}


\begin{savenotes}\sphinxattablestart
\sphinxthistablewithglobalstyle
\centering
\begin{tabulary}{\linewidth}[t]{TT}
\sphinxtoprule
\sphinxtableatstartofbodyhook
\sphinxAtStartPar
Blank
&
\sphinxAtStartPar
Number of the scenario
\\
\sphinxhline
\sphinxAtStartPar
Scenario
&
\sphinxAtStartPar
name of the scenario
\\
\sphinxhline
\sphinxAtStartPar
Content
&
\sphinxAtStartPar
is content send (geometry, views, etc.) you could just send settings (time, position etc.)
\\
\sphinxhline
\sphinxAtStartPar
Clear
&
\sphinxAtStartPar
Clears the channel before uploading new geometry
\\
\sphinxhline
\sphinxAtStartPar
Load
&
\sphinxAtStartPar
loading from the channel
\\
\sphinxhline
\sphinxAtStartPar
Duration
&
\sphinxAtStartPar
of the scenario
\\
\sphinxbottomrule
\end{tabulary}
\sphinxtableafterendhook\par
\sphinxattableend\end{savenotes}

\sphinxAtStartPar
The saved content is stored in the component, be aware that huge amounts of geometry can make your .gh file very heavy and slow.

\sphinxAtStartPar
The current \sphinxcode{\sphinxupquote{best practice for heavy geometry}} is to upload it to the server via the cloud manager in the save component and then command the
download via the publishControl \textendash{}\textgreater{} File settings

\sphinxAtStartPar
\sphinxstylestrong{Examples}

\sphinxAtStartPar
You have some geometry (a building) and want to publish or download from the server (1), then walk through it, change the time of the day (2) and
continue your tour via a series of pre defined views (3\sphinxhyphen{}4).
Instead of setting everything live during your presentation, you define one position after the other and save
them as individual scenarios. You then can switch through them during your presentation more easily.

\sphinxAtStartPar
\sphinxstylestrong{1)}

\noindent\sphinxincludegraphics{{1}.png}

\sphinxAtStartPar
\sphinxstylestrong{2)}

\noindent\sphinxincludegraphics{{2}.png}

\sphinxAtStartPar
\sphinxstylestrong{3\sphinxhyphen{}4)}

\noindent\sphinxincludegraphics{{3}.png}

\noindent\sphinxincludegraphics{{4}.png}

\sphinxstepscope


\subsubsection{PublishView}
\label{\detokenize{tutorial/grashopper/documentation_rst/07_publish_View:publishview}}\label{\detokenize{tutorial/grashopper/documentation_rst/07_publish_View::doc}}
\noindent\sphinxincludegraphics[scale=0.8]{{Publish_view}.png}
\begin{itemize}
\item {} 
\sphinxAtStartPar
there are two ways to import views:
\begin{itemize}
\item {} 
\sphinxAtStartPar
grasshopper via the component input

\item {} 
\sphinxAtStartPar
Saved Rhino views are accessible in the component menu

\end{itemize}

\end{itemize}

\sphinxAtStartPar
\sphinxstylestrong{Input}


\begin{savenotes}\sphinxattablestart
\sphinxthistablewithglobalstyle
\centering
\begin{tabulary}{\linewidth}[t]{TTT}
\sphinxtoprule
\sphinxstyletheadfamily 
\sphinxAtStartPar
Name
&\sphinxstyletheadfamily 
\sphinxAtStartPar
Description
&\sphinxstyletheadfamily 
\sphinxAtStartPar
Type
\\
\sphinxmidrule
\sphinxtableatstartofbodyhook
\sphinxAtStartPar
Connection
&
\sphinxAtStartPar
Link with the Connect component
&
\sphinxAtStartPar
Connect
\\
\sphinxhline
\sphinxAtStartPar
Camera Planes
&
\sphinxAtStartPar
Planes to define viewpoints
&
\sphinxAtStartPar
Plane
\\
\sphinxhline
\sphinxAtStartPar
Field of view
&
\sphinxAtStartPar
Size of view
&
\sphinxAtStartPar
Number
\\
\sphinxhline
\sphinxAtStartPar
Index
&
\sphinxAtStartPar
To switch between views
&
\sphinxAtStartPar
Number
\\
\sphinxbottomrule
\end{tabulary}
\sphinxtableafterendhook\par
\sphinxattableend\end{savenotes}

\sphinxAtStartPar
\sphinxstylestrong{Output}


\begin{savenotes}\sphinxattablestart
\sphinxthistablewithglobalstyle
\centering
\begin{tabulary}{\linewidth}[t]{TTT}
\sphinxtoprule
\sphinxstyletheadfamily 
\sphinxAtStartPar
Name
&\sphinxstyletheadfamily 
\sphinxAtStartPar
Description
&\sphinxstyletheadfamily 
\sphinxAtStartPar
Type
\\
\sphinxmidrule
\sphinxtableatstartofbodyhook
\sphinxAtStartPar
Log
&
\sphinxAtStartPar
Documents changes \& Data send
&
\sphinxAtStartPar
Text
\\
\sphinxhline
\sphinxAtStartPar
Save
&
\sphinxAtStartPar
Connect to SaveContent for saving
&
\sphinxAtStartPar
Radii content
\\
\sphinxbottomrule
\end{tabulary}
\sphinxtableafterendhook\par
\sphinxattableend\end{savenotes}

\sphinxAtStartPar
\sphinxstylestrong{Menu}


\begin{savenotes}\sphinxattablestart
\sphinxthistablewithglobalstyle
\centering
\begin{tabulary}{\linewidth}[t]{TT}
\sphinxtoprule
\sphinxtableatstartofbodyhook
\sphinxAtStartPar
View
&
\sphinxAtStartPar
Duration Speed to switch between views
\\
\sphinxhline
\sphinxAtStartPar
Active view
&
\sphinxAtStartPar
Rhino Views
\\
\sphinxbottomrule
\end{tabulary}
\sphinxtableafterendhook\par
\sphinxattableend\end{savenotes}

\sphinxstepscope


\subsubsection{PublishPointcloud}
\label{\detokenize{tutorial/grashopper/documentation_rst/08_publish_Pointcloud:publishpointcloud}}\label{\detokenize{tutorial/grashopper/documentation_rst/08_publish_Pointcloud::doc}}
\noindent\sphinxincludegraphics[scale=0.9]{{Publish_Pointclouds}.png}

\sphinxAtStartPar
\sphinxstylestrong{Input}


\begin{savenotes}\sphinxattablestart
\sphinxthistablewithglobalstyle
\centering
\begin{tabulary}{\linewidth}[t]{TTT}
\sphinxtoprule
\sphinxstyletheadfamily 
\sphinxAtStartPar
Name
&\sphinxstyletheadfamily 
\sphinxAtStartPar
Description
&\sphinxstyletheadfamily 
\sphinxAtStartPar
Type
\\
\sphinxmidrule
\sphinxtableatstartofbodyhook
\sphinxAtStartPar
Connection
&
\sphinxAtStartPar
Link with the Connect component
&
\sphinxAtStartPar
Connect
\\
\sphinxhline
\sphinxAtStartPar
Point Cloud
&
\sphinxAtStartPar
Input for a Point cloud
&
\sphinxAtStartPar
Point cloud
\\
\sphinxbottomrule
\end{tabulary}
\sphinxtableafterendhook\par
\sphinxattableend\end{savenotes}
\begin{itemize}
\item {} 
\sphinxAtStartPar
Point clouds have a tendency to be strain on your pc, in those cases see more under the tool section of radii

\end{itemize}

\sphinxAtStartPar
\sphinxstylestrong{Output}


\begin{savenotes}\sphinxattablestart
\sphinxthistablewithglobalstyle
\centering
\begin{tabulary}{\linewidth}[t]{TTT}
\sphinxtoprule
\sphinxstyletheadfamily 
\sphinxAtStartPar
Name
&\sphinxstyletheadfamily 
\sphinxAtStartPar
Description
&\sphinxstyletheadfamily 
\sphinxAtStartPar
Type
\\
\sphinxmidrule
\sphinxtableatstartofbodyhook
\sphinxAtStartPar
Log
&
\sphinxAtStartPar
Documents changes \& Data send
&
\sphinxAtStartPar
Text
\\
\sphinxhline
\sphinxAtStartPar
Save
&
\sphinxAtStartPar
Connect to SaveContent for saving
&
\sphinxAtStartPar
Radii content
\\
\sphinxbottomrule
\end{tabulary}
\sphinxtableafterendhook\par
\sphinxattableend\end{savenotes}

\sphinxstepscope


\subsubsection{PublishAnimation}
\label{\detokenize{tutorial/grashopper/documentation_rst/09_publish_Animation:publishanimation}}\label{\detokenize{tutorial/grashopper/documentation_rst/09_publish_Animation::doc}}
\noindent\sphinxincludegraphics[scale=0.9]{{Publish_animation}.png}

\sphinxAtStartPar
\sphinxstylestrong{Input}


\begin{savenotes}\sphinxattablestart
\sphinxthistablewithglobalstyle
\centering
\begin{tabulary}{\linewidth}[t]{TTT}
\sphinxtoprule
\sphinxstyletheadfamily 
\sphinxAtStartPar
Name
&\sphinxstyletheadfamily 
\sphinxAtStartPar
Description
&\sphinxstyletheadfamily 
\sphinxAtStartPar
Type
\\
\sphinxmidrule
\sphinxtableatstartofbodyhook
\sphinxAtStartPar
Connection
&
\sphinxAtStartPar
Link with the Connect component
&
\sphinxAtStartPar
Connect
\\
\sphinxhline
\sphinxAtStartPar
Animation Planes
&
\sphinxAtStartPar
Along the path you want to animate along
&
\sphinxAtStartPar
Planes
\\
\sphinxhline
\sphinxAtStartPar
Content
&
\sphinxAtStartPar
Geometry you want to animate
&
\sphinxAtStartPar
Save of Publish Geometry component
\\
\sphinxbottomrule
\end{tabulary}
\sphinxtableafterendhook\par
\sphinxattableend\end{savenotes}

\sphinxAtStartPar
\sphinxstylestrong{Output}


\begin{savenotes}\sphinxattablestart
\sphinxthistablewithglobalstyle
\centering
\begin{tabulary}{\linewidth}[t]{TTT}
\sphinxtoprule
\sphinxstyletheadfamily 
\sphinxAtStartPar
Name
&\sphinxstyletheadfamily 
\sphinxAtStartPar
Description
&\sphinxstyletheadfamily 
\sphinxAtStartPar
Type
\\
\sphinxmidrule
\sphinxtableatstartofbodyhook
\sphinxAtStartPar
Log
&
\sphinxAtStartPar
Documents changes \& Data send
&
\sphinxAtStartPar
Text
\\
\sphinxhline
\sphinxAtStartPar
Save
&
\sphinxAtStartPar
Connect to SaveContent for saving
&
\sphinxAtStartPar
Radii content
\\
\sphinxbottomrule
\end{tabulary}
\sphinxtableafterendhook\par
\sphinxattableend\end{savenotes}

\sphinxAtStartPar
\sphinxstylestrong{Menu}


\begin{savenotes}\sphinxattablestart
\sphinxthistablewithglobalstyle
\centering
\begin{tabulary}{\linewidth}[t]{TT}
\sphinxtoprule
\sphinxtableatstartofbodyhook
\sphinxAtStartPar
Animation
&
\sphinxAtStartPar
Title Name of your Animation
\\
\sphinxhline
\sphinxAtStartPar
Animation
&
\sphinxAtStartPar
Duration Speed: higher number = quicker
\\
\sphinxhline
\sphinxAtStartPar
Animation
&
\sphinxAtStartPar
Behavior Play, Replay, Reverse, Return
\\
\sphinxbottomrule
\end{tabulary}
\sphinxtableafterendhook\par
\sphinxattableend\end{savenotes}

\sphinxstepscope


\subsubsection{PublishCurve}
\label{\detokenize{tutorial/grashopper/documentation_rst/10_publish_Curve:publishcurve}}\label{\detokenize{tutorial/grashopper/documentation_rst/10_publish_Curve::doc}}
\noindent\sphinxincludegraphics{{Publish_curve}.png}

\sphinxAtStartPar
\sphinxstylestrong{Input}


\begin{savenotes}\sphinxattablestart
\sphinxthistablewithglobalstyle
\centering
\begin{tabulary}{\linewidth}[t]{TTT}
\sphinxtoprule
\sphinxstyletheadfamily 
\sphinxAtStartPar
Name
&\sphinxstyletheadfamily 
\sphinxAtStartPar
Description
&\sphinxstyletheadfamily 
\sphinxAtStartPar
Type
\\
\sphinxmidrule
\sphinxtableatstartofbodyhook
\sphinxAtStartPar
Connection
&
\sphinxAtStartPar
Link with the Connect component
&
\sphinxAtStartPar
Connect
\\
\sphinxhline
\sphinxAtStartPar
Curve
&
\sphinxAtStartPar
the curves to publish
&
\sphinxAtStartPar
Curve
\\
\sphinxhline
\sphinxAtStartPar
MinEdge
&
\sphinxAtStartPar
Min length of segments
&
\sphinxAtStartPar
Number
\\
\sphinxhline
\sphinxAtStartPar
MaxEdge
&
\sphinxAtStartPar
Max length of segments
&
\sphinxAtStartPar
Number
\\
\sphinxbottomrule
\end{tabulary}
\sphinxtableafterendhook\par
\sphinxattableend\end{savenotes}

\sphinxAtStartPar
\sphinxstylestrong{Output}


\begin{savenotes}\sphinxattablestart
\sphinxthistablewithglobalstyle
\centering
\begin{tabulary}{\linewidth}[t]{TTT}
\sphinxtoprule
\sphinxstyletheadfamily 
\sphinxAtStartPar
Name
&\sphinxstyletheadfamily 
\sphinxAtStartPar
Description
&\sphinxstyletheadfamily 
\sphinxAtStartPar
Type
\\
\sphinxmidrule
\sphinxtableatstartofbodyhook
\sphinxAtStartPar
Log
&
\sphinxAtStartPar
Documents changes \& Data send
&
\sphinxAtStartPar
Text
\\
\sphinxhline
\sphinxAtStartPar
Save
&
\sphinxAtStartPar
Connect to SaveContent for saving
&
\sphinxAtStartPar
Radii content
\\
\sphinxbottomrule
\end{tabulary}
\sphinxtableafterendhook\par
\sphinxattableend\end{savenotes}

\sphinxstepscope


\subsubsection{PublishMessage}
\label{\detokenize{tutorial/grashopper/documentation_rst/11_publish_Message:publishmessage}}\label{\detokenize{tutorial/grashopper/documentation_rst/11_publish_Message::doc}}
\noindent\sphinxincludegraphics{{Publish_message}.png}

\sphinxAtStartPar
\sphinxstylestrong{Input}


\begin{savenotes}\sphinxattablestart
\sphinxthistablewithglobalstyle
\centering
\begin{tabulary}{\linewidth}[t]{TTT}
\sphinxtoprule
\sphinxstyletheadfamily 
\sphinxAtStartPar
Name
&\sphinxstyletheadfamily 
\sphinxAtStartPar
Description
&\sphinxstyletheadfamily 
\sphinxAtStartPar
Type
\\
\sphinxmidrule
\sphinxtableatstartofbodyhook
\sphinxAtStartPar
Connection
&
\sphinxAtStartPar
Link with the Connect component
&
\sphinxAtStartPar
Connection
\\
\sphinxbottomrule
\end{tabulary}
\sphinxtableafterendhook\par
\sphinxattableend\end{savenotes}

\sphinxAtStartPar
\sphinxstylestrong{Menu}


\begin{savenotes}\sphinxattablestart
\sphinxthistablewithglobalstyle
\centering
\begin{tabulary}{\linewidth}[t]{TT}
\sphinxtoprule
\sphinxtableatstartofbodyhook
\sphinxAtStartPar
Send Massage to channel
&
\sphinxAtStartPar
Type a message and press enter to send all viewers that are connected
\\
\sphinxbottomrule
\end{tabulary}
\sphinxtableafterendhook\par
\sphinxattableend\end{savenotes}

\sphinxAtStartPar
\sphinxstylestrong{Save Content}

\sphinxstepscope


\subsubsection{SaveContent}
\label{\detokenize{tutorial/grashopper/documentation_rst/12_SaveContent:savecontent}}\label{\detokenize{tutorial/grashopper/documentation_rst/12_SaveContent::doc}}
\noindent\sphinxincludegraphics{{Save_Content}.png}

\sphinxAtStartPar
\sphinxstylestrong{Input}


\begin{savenotes}\sphinxattablestart
\sphinxthistablewithglobalstyle
\centering
\begin{tabulary}{\linewidth}[t]{TTT}
\sphinxtoprule
\sphinxstyletheadfamily 
\sphinxAtStartPar
Name
&\sphinxstyletheadfamily 
\sphinxAtStartPar
Description
&\sphinxstyletheadfamily 
\sphinxAtStartPar
Type
\\
\sphinxmidrule
\sphinxtableatstartofbodyhook
\sphinxAtStartPar
Connection
&
\sphinxAtStartPar
Link with the Connect component
&
\sphinxAtStartPar
Connection
\\
\sphinxhline
\sphinxAtStartPar
Content
&
\sphinxAtStartPar
To be included in the save
&
\sphinxAtStartPar
Radii content
\\
\sphinxbottomrule
\end{tabulary}
\sphinxtableafterendhook\par
\sphinxattableend\end{savenotes}

\sphinxAtStartPar
\sphinxstylestrong{Menu}


\begin{savenotes}\sphinxattablestart
\sphinxthistablewithglobalstyle
\centering
\begin{tabulary}{\linewidth}[t]{TT}
\sphinxtoprule
\sphinxtableatstartofbodyhook
\sphinxAtStartPar
Save
&
\sphinxAtStartPar
Save a Radii file locally
\\
\sphinxhline
\sphinxAtStartPar
Cloud Manager
&
\sphinxAtStartPar
Save to the connected channel
\\
\sphinxbottomrule
\end{tabulary}
\sphinxtableafterendhook\par
\sphinxattableend\end{savenotes}

\noindent\sphinxincludegraphics[scale=1.0]{{Save_Content2}.png}
\begin{itemize}
\item {} 
\sphinxAtStartPar
Cloud content can also be loaded via the scenario manager

\end{itemize}

\sphinxAtStartPar
\sphinxstylestrong{Subscribe Components}

\sphinxstepscope


\subsubsection{SubscribeCurve}
\label{\detokenize{tutorial/grashopper/documentation_rst/13_SubscribeCurve:subscribecurve}}\label{\detokenize{tutorial/grashopper/documentation_rst/13_SubscribeCurve::doc}}
\noindent\sphinxincludegraphics[scale=0.9]{{Sub_curve}.png}

\sphinxAtStartPar
\sphinxstylestrong{Input}


\begin{savenotes}\sphinxattablestart
\sphinxthistablewithglobalstyle
\centering
\begin{tabulary}{\linewidth}[t]{TTT}
\sphinxtoprule
\sphinxstyletheadfamily 
\sphinxAtStartPar
Name
&\sphinxstyletheadfamily 
\sphinxAtStartPar
Description
&\sphinxstyletheadfamily 
\sphinxAtStartPar
Type
\\
\sphinxmidrule
\sphinxtableatstartofbodyhook
\sphinxAtStartPar
Connection
&
\sphinxAtStartPar
Link with the Connect component
&
\sphinxAtStartPar
Connection
\\
\sphinxhline
\sphinxAtStartPar
Filter
&
\sphinxAtStartPar
Filter own publication/broadcast
&
\sphinxAtStartPar
Boolean
\\
\sphinxhline
\sphinxAtStartPar
Subscribe
&
\sphinxAtStartPar
Toggle the subscription
&
\sphinxAtStartPar
Boolean
\\
\sphinxbottomrule
\end{tabulary}
\sphinxtableafterendhook\par
\sphinxattableend\end{savenotes}

\sphinxAtStartPar
\sphinxstylestrong{Output}


\begin{savenotes}\sphinxattablestart
\sphinxthistablewithglobalstyle
\centering
\begin{tabulary}{\linewidth}[t]{TTT}
\sphinxtoprule
\sphinxstyletheadfamily 
\sphinxAtStartPar
Name
&\sphinxstyletheadfamily 
\sphinxAtStartPar
Description
&\sphinxstyletheadfamily 
\sphinxAtStartPar
Type
\\
\sphinxmidrule
\sphinxtableatstartofbodyhook
\sphinxAtStartPar
Log
&
\sphinxAtStartPar
Documents changes \& Data send
&
\sphinxAtStartPar
Text
\\
\sphinxhline
\sphinxAtStartPar
Curve
&
\sphinxAtStartPar
Element to work with
&
\sphinxAtStartPar
Curve
\\
\sphinxbottomrule
\end{tabulary}
\sphinxtableafterendhook\par
\sphinxattableend\end{savenotes}

\sphinxstepscope


\subsubsection{SubscribeMessages}
\label{\detokenize{tutorial/grashopper/documentation_rst/14_SubscribeMessages:subscribemessages}}\label{\detokenize{tutorial/grashopper/documentation_rst/14_SubscribeMessages::doc}}
\noindent\sphinxincludegraphics{{Sub_message}.png}

\sphinxAtStartPar
\sphinxstylestrong{Input}


\begin{savenotes}\sphinxattablestart
\sphinxthistablewithglobalstyle
\centering
\begin{tabulary}{\linewidth}[t]{TTT}
\sphinxtoprule
\sphinxstyletheadfamily 
\sphinxAtStartPar
Name
&\sphinxstyletheadfamily 
\sphinxAtStartPar
Description
&\sphinxstyletheadfamily 
\sphinxAtStartPar
Type
\\
\sphinxmidrule
\sphinxtableatstartofbodyhook
\sphinxAtStartPar
Connection
&
\sphinxAtStartPar
Link with the Connect component
&
\sphinxAtStartPar
Connection
\\
\sphinxhline
\sphinxAtStartPar
Subscribe
&
\sphinxAtStartPar
Toggle the subscription
&
\sphinxAtStartPar
Boolean
\\
\sphinxbottomrule
\end{tabulary}
\sphinxtableafterendhook\par
\sphinxattableend\end{savenotes}

\sphinxAtStartPar
\sphinxstylestrong{Output}


\begin{savenotes}\sphinxattablestart
\sphinxthistablewithglobalstyle
\centering
\begin{tabulary}{\linewidth}[t]{TTT}
\sphinxtoprule
\sphinxstyletheadfamily 
\sphinxAtStartPar
Name
&\sphinxstyletheadfamily 
\sphinxAtStartPar
Description
&\sphinxstyletheadfamily 
\sphinxAtStartPar
Type
\\
\sphinxmidrule
\sphinxtableatstartofbodyhook
\sphinxAtStartPar
Messages
&
\sphinxAtStartPar
Connect to a Notepad to observe
&
\sphinxAtStartPar
Text
\\
\sphinxbottomrule
\end{tabulary}
\sphinxtableafterendhook\par
\sphinxattableend\end{savenotes}

\sphinxstepscope


\subsubsection{SubscribePointCloud}
\label{\detokenize{tutorial/grashopper/documentation_rst/15_SubscribePointCloud:subscribepointcloud}}\label{\detokenize{tutorial/grashopper/documentation_rst/15_SubscribePointCloud::doc}}
\noindent\sphinxincludegraphics{{Pointcloud}.png}

\sphinxAtStartPar
\sphinxstylestrong{Input}


\begin{savenotes}\sphinxattablestart
\sphinxthistablewithglobalstyle
\centering
\begin{tabulary}{\linewidth}[t]{TTT}
\sphinxtoprule
\sphinxstyletheadfamily 
\sphinxAtStartPar
Name
&\sphinxstyletheadfamily 
\sphinxAtStartPar
Description
&\sphinxstyletheadfamily 
\sphinxAtStartPar
Type
\\
\sphinxmidrule
\sphinxtableatstartofbodyhook
\sphinxAtStartPar
Connection
&
\sphinxAtStartPar
Link with the Connect component
&
\sphinxAtStartPar
Connection
\\
\sphinxhline
\sphinxAtStartPar
Filter
&
\sphinxAtStartPar
Filter own publication/broadcast
&
\sphinxAtStartPar
Boolean
\\
\sphinxhline
\sphinxAtStartPar
Subscribe
&
\sphinxAtStartPar
Toggle the subscription
&
\sphinxAtStartPar
Boolean
\\
\sphinxbottomrule
\end{tabulary}
\sphinxtableafterendhook\par
\sphinxattableend\end{savenotes}

\sphinxAtStartPar
\sphinxstylestrong{Output}


\begin{savenotes}\sphinxattablestart
\sphinxthistablewithglobalstyle
\centering
\begin{tabulary}{\linewidth}[t]{TTT}
\sphinxtoprule
\sphinxstyletheadfamily 
\sphinxAtStartPar
Name
&\sphinxstyletheadfamily 
\sphinxAtStartPar
Description
&\sphinxstyletheadfamily 
\sphinxAtStartPar
Type
\\
\sphinxmidrule
\sphinxtableatstartofbodyhook
\sphinxAtStartPar
Point Cloud
&
\sphinxAtStartPar
Element to work with
&
\sphinxAtStartPar
Point Cloud
\\
\sphinxbottomrule
\end{tabulary}
\sphinxtableafterendhook\par
\sphinxattableend\end{savenotes}

\sphinxstepscope


\subsubsection{SubscribeGeometry}
\label{\detokenize{tutorial/grashopper/documentation_rst/16_SubscribeGeometry:subscribegeometry}}\label{\detokenize{tutorial/grashopper/documentation_rst/16_SubscribeGeometry::doc}}
\noindent\sphinxincludegraphics{{Sub_geometry}.png}

\sphinxAtStartPar
\sphinxstylestrong{Input}


\begin{savenotes}\sphinxattablestart
\sphinxthistablewithglobalstyle
\centering
\begin{tabulary}{\linewidth}[t]{TTT}
\sphinxtoprule
\sphinxstyletheadfamily 
\sphinxAtStartPar
Name
&\sphinxstyletheadfamily 
\sphinxAtStartPar
Description
&\sphinxstyletheadfamily 
\sphinxAtStartPar
Type
\\
\sphinxmidrule
\sphinxtableatstartofbodyhook
\sphinxAtStartPar
Connection
&
\sphinxAtStartPar
Link with the Connect component
&
\sphinxAtStartPar
Connection
\\
\sphinxhline
\sphinxAtStartPar
Filter
&
\sphinxAtStartPar
Filter own publication/broadcast
&
\sphinxAtStartPar
Boolean
\\
\sphinxhline
\sphinxAtStartPar
Subscribe
&
\sphinxAtStartPar
Toggle the subscription
&
\sphinxAtStartPar
Boolean
\\
\sphinxbottomrule
\end{tabulary}
\sphinxtableafterendhook\par
\sphinxattableend\end{savenotes}

\sphinxAtStartPar
\sphinxstylestrong{Output}


\begin{savenotes}\sphinxattablestart
\sphinxthistablewithglobalstyle
\centering
\begin{tabulary}{\linewidth}[t]{TTT}
\sphinxtoprule
\sphinxstyletheadfamily 
\sphinxAtStartPar
Name
&\sphinxstyletheadfamily 
\sphinxAtStartPar
Description
&\sphinxstyletheadfamily 
\sphinxAtStartPar
Type
\\
\sphinxmidrule
\sphinxtableatstartofbodyhook
\sphinxAtStartPar
Log
&
\sphinxAtStartPar
Documents changes \& Data send
&
\sphinxAtStartPar
Text
\\
\sphinxhline
\sphinxAtStartPar
Geometry
&
\sphinxAtStartPar
Element to work with
&
\sphinxAtStartPar
Geometry
\\
\sphinxbottomrule
\end{tabulary}
\sphinxtableafterendhook\par
\sphinxattableend\end{savenotes}

\sphinxstepscope


\subsubsection{SubscribeParameter}
\label{\detokenize{tutorial/grashopper/documentation_rst/17_SubscribeParameter:subscribeparameter}}\label{\detokenize{tutorial/grashopper/documentation_rst/17_SubscribeParameter::doc}}
\noindent{\sphinxincludegraphics{{Sub_parameter}.png}\hspace*{\fill}}

\sphinxAtStartPar
Grashopper component

\noindent\sphinxincludegraphics{{Sub_parameter22}.png}

\sphinxAtStartPar
Radii viewer counterpart

\sphinxAtStartPar
\sphinxstylestrong{Input}


\begin{savenotes}\sphinxattablestart
\sphinxthistablewithglobalstyle
\centering
\begin{tabulary}{\linewidth}[t]{TTT}
\sphinxtoprule
\sphinxstyletheadfamily 
\sphinxAtStartPar
Name
&\sphinxstyletheadfamily 
\sphinxAtStartPar
Description
&\sphinxstyletheadfamily 
\sphinxAtStartPar
Type
\\
\sphinxmidrule
\sphinxtableatstartofbodyhook
\sphinxAtStartPar
Connection
&
\sphinxAtStartPar
Link with the Connect component
&
\sphinxAtStartPar
Connection
\\
\sphinxhline
\sphinxAtStartPar
Subscribe
&
\sphinxAtStartPar
Toggle the subscription
&
\sphinxAtStartPar
Boolean
\\
\sphinxbottomrule
\end{tabulary}
\sphinxtableafterendhook\par
\sphinxattableend\end{savenotes}

\sphinxAtStartPar
\sphinxstylestrong{Output}


\begin{savenotes}\sphinxattablestart
\sphinxthistablewithglobalstyle
\centering
\begin{tabulary}{\linewidth}[t]{TTT}
\sphinxtoprule
\sphinxstyletheadfamily 
\sphinxAtStartPar
Name
&\sphinxstyletheadfamily 
\sphinxAtStartPar
Description
&\sphinxstyletheadfamily 
\sphinxAtStartPar
Type
\\
\sphinxmidrule
\sphinxtableatstartofbodyhook
\sphinxAtStartPar
Log
&
\sphinxAtStartPar
Documents changes \& Data send
&
\sphinxAtStartPar
Text
\\
\sphinxhline
\sphinxAtStartPar
Parameter/Boolean
&
\sphinxAtStartPar
Parameter/Boolean from Radii Viewer
&
\sphinxAtStartPar
Boolean/Number
\\
\sphinxbottomrule
\end{tabulary}
\sphinxtableafterendhook\par
\sphinxattableend\end{savenotes}
\begin{itemize}
\item {} 
\sphinxAtStartPar
The more parameters you define in the viewer the more will be on this component

\item {} 
\sphinxAtStartPar
the number will be between 0 to 1, you can remap this to any other range

\end{itemize}

\sphinxstepscope


\subsubsection{SubscribeUser}
\label{\detokenize{tutorial/grashopper/documentation_rst/18_SubscribeUser:subscribeuser}}\label{\detokenize{tutorial/grashopper/documentation_rst/18_SubscribeUser::doc}}
\noindent\sphinxincludegraphics{{Sub_user}.png}

\sphinxAtStartPar
\sphinxstylestrong{Input}


\begin{savenotes}\sphinxattablestart
\sphinxthistablewithglobalstyle
\centering
\begin{tabulary}{\linewidth}[t]{TTT}
\sphinxtoprule
\sphinxstyletheadfamily 
\sphinxAtStartPar
Name
&\sphinxstyletheadfamily 
\sphinxAtStartPar
Description
&\sphinxstyletheadfamily 
\sphinxAtStartPar
Type
\\
\sphinxmidrule
\sphinxtableatstartofbodyhook
\sphinxAtStartPar
Connection
&
\sphinxAtStartPar
Link with the Connect component
&
\sphinxAtStartPar
Connection
\\
\sphinxhline
\sphinxAtStartPar
Follow
&
\sphinxAtStartPar
Follow a user (select in the menu)
&
\sphinxAtStartPar
Boolean
\\
\sphinxhline
\sphinxAtStartPar
Filter
&
\sphinxAtStartPar
Filter own publication/broadcast
&
\sphinxAtStartPar
Boolean
\\
\sphinxhline
\sphinxAtStartPar
Subscribe
&
\sphinxAtStartPar
Toggle the subscription
&
\sphinxAtStartPar
Boolean
\\
\sphinxbottomrule
\end{tabulary}
\sphinxtableafterendhook\par
\sphinxattableend\end{savenotes}

\sphinxAtStartPar
\sphinxstylestrong{Output}


\begin{savenotes}\sphinxattablestart
\sphinxthistablewithglobalstyle
\centering
\begin{tabulary}{\linewidth}[t]{TTT}
\sphinxtoprule
\sphinxstyletheadfamily 
\sphinxAtStartPar
Name
&\sphinxstyletheadfamily 
\sphinxAtStartPar
Description
&\sphinxstyletheadfamily 
\sphinxAtStartPar
Type
\\
\sphinxmidrule
\sphinxtableatstartofbodyhook
\sphinxAtStartPar
User
&
\sphinxAtStartPar
User List
&
\sphinxAtStartPar
Text
\\
\sphinxhline
\sphinxAtStartPar
Position
&
\sphinxAtStartPar
Coordinates
&
\sphinxAtStartPar
Point
\\
\sphinxhline
\sphinxAtStartPar
Forward
&
\sphinxAtStartPar
Direction of View
&
\sphinxAtStartPar
Vector
\\
\sphinxhline
\sphinxAtStartPar
Upward
&
\sphinxAtStartPar
Direction upwards
&
\sphinxAtStartPar
Vector
\\
\sphinxhline
\sphinxAtStartPar
FOV
&
\sphinxAtStartPar
Field of View
&
\sphinxAtStartPar
Number
\\
\sphinxhline
\sphinxAtStartPar
Right Position
&
\sphinxAtStartPar
Right VR Position
&
\sphinxAtStartPar
Point
\\
\sphinxhline
\sphinxAtStartPar
Right Forward
&
\sphinxAtStartPar
Direction forward
&
\sphinxAtStartPar
Vector
\\
\sphinxhline
\sphinxAtStartPar
Right Upward
&
\sphinxAtStartPar
Direction upwards
&
\sphinxAtStartPar
Vector
\\
\sphinxhline
\sphinxAtStartPar
Left Position
&
\sphinxAtStartPar
Left VR Position
&
\sphinxAtStartPar
Point
\\
\sphinxhline
\sphinxAtStartPar
Left Forward
&
\sphinxAtStartPar
Direction forward
&
\sphinxAtStartPar
Vector
\\
\sphinxhline
\sphinxAtStartPar
Left Upward
&
\sphinxAtStartPar
Direction upwards
&
\sphinxAtStartPar
Vector
\\
\sphinxbottomrule
\end{tabulary}
\sphinxtableafterendhook\par
\sphinxattableend\end{savenotes}

\sphinxAtStartPar
\sphinxstylestrong{Menu}


\begin{savenotes}\sphinxattablestart
\sphinxthistablewithglobalstyle
\centering
\begin{tabulary}{\linewidth}[t]{TT}
\sphinxtoprule
\sphinxtableatstartofbodyhook
\sphinxAtStartPar
Users
&
\sphinxAtStartPar
List of users to select
\\
\sphinxbottomrule
\end{tabulary}
\sphinxtableafterendhook\par
\sphinxattableend\end{savenotes}

\sphinxAtStartPar
\sphinxstylestrong{Tool Components \& Tips}

\sphinxstepscope


\subsubsection{PointCloudReduce \& SubsetPointCloud}
\label{\detokenize{tutorial/grashopper/documentation_rst/19_Tools_PointCloudResource_subset:pointcloudreduce-subsetpointcloud}}\label{\detokenize{tutorial/grashopper/documentation_rst/19_Tools_PointCloudResource_subset::doc}}

\paragraph{ReducePointCloud}
\label{\detokenize{tutorial/grashopper/documentation_rst/19_Tools_PointCloudResource_subset:reducepointcloud}}
\noindent\sphinxincludegraphics{{Pointcloudreduce}.png}

\sphinxAtStartPar
\sphinxstylestrong{Input}


\begin{savenotes}\sphinxattablestart
\sphinxthistablewithglobalstyle
\centering
\begin{tabulary}{\linewidth}[t]{TTT}
\sphinxtoprule
\sphinxstyletheadfamily 
\sphinxAtStartPar
Name
&\sphinxstyletheadfamily 
\sphinxAtStartPar
Description
&\sphinxstyletheadfamily 
\sphinxAtStartPar
Type
\\
\sphinxmidrule
\sphinxtableatstartofbodyhook
\sphinxAtStartPar
Point Cloud
&
\sphinxAtStartPar
Point Cloud to Reduce
&
\sphinxAtStartPar
Point Cloud
\\
\sphinxhline
\sphinxAtStartPar
Fraction
&
\sphinxAtStartPar
Fraction of Points to remain
&
\sphinxAtStartPar
Number
\\
\sphinxbottomrule
\end{tabulary}
\sphinxtableafterendhook\par
\sphinxattableend\end{savenotes}

\sphinxAtStartPar
\sphinxstylestrong{Output}


\begin{savenotes}\sphinxattablestart
\sphinxthistablewithglobalstyle
\centering
\begin{tabulary}{\linewidth}[t]{TTT}
\sphinxtoprule
\sphinxstyletheadfamily 
\sphinxAtStartPar
Name
&\sphinxstyletheadfamily 
\sphinxAtStartPar
Description
&\sphinxstyletheadfamily 
\sphinxAtStartPar
Type
\\
\sphinxmidrule
\sphinxtableatstartofbodyhook
\sphinxAtStartPar
Point Cloud
&
\sphinxAtStartPar
Reduced Point Cloud
&
\sphinxAtStartPar
Point Cloud
\\
\sphinxbottomrule
\end{tabulary}
\sphinxtableafterendhook\par
\sphinxattableend\end{savenotes}


\paragraph{SubsetPointCloud}
\label{\detokenize{tutorial/grashopper/documentation_rst/19_Tools_PointCloudResource_subset:subsetpointcloud}}
\noindent\sphinxincludegraphics{{Pointcloudbox}.png}

\sphinxAtStartPar
\sphinxstylestrong{Input}


\begin{savenotes}\sphinxattablestart
\sphinxthistablewithglobalstyle
\centering
\begin{tabulary}{\linewidth}[t]{TTT}
\sphinxtoprule
\sphinxstyletheadfamily 
\sphinxAtStartPar
Name
&\sphinxstyletheadfamily 
\sphinxAtStartPar
Description
&\sphinxstyletheadfamily 
\sphinxAtStartPar
Type
\\
\sphinxmidrule
\sphinxtableatstartofbodyhook
\sphinxAtStartPar
Point Cloud
&
\sphinxAtStartPar
Point Cloud to Reduce
&
\sphinxAtStartPar
Point Cloud
\\
\sphinxhline
\sphinxAtStartPar
Selection box
&
\sphinxAtStartPar
Volume you want to keep
&
\sphinxAtStartPar
Box
\\
\sphinxbottomrule
\end{tabulary}
\sphinxtableafterendhook\par
\sphinxattableend\end{savenotes}

\sphinxAtStartPar
\sphinxstylestrong{Output}


\begin{savenotes}\sphinxattablestart
\sphinxthistablewithglobalstyle
\centering
\begin{tabulary}{\linewidth}[t]{TTT}
\sphinxtoprule
\sphinxstyletheadfamily 
\sphinxAtStartPar
Name
&\sphinxstyletheadfamily 
\sphinxAtStartPar
Description
&\sphinxstyletheadfamily 
\sphinxAtStartPar
Type
\\
\sphinxmidrule
\sphinxtableatstartofbodyhook
\sphinxAtStartPar
Point Cloud
&
\sphinxAtStartPar
Selection of Point Cloud
&
\sphinxAtStartPar
Point Cloud
\\
\sphinxbottomrule
\end{tabulary}
\sphinxtableafterendhook\par
\sphinxattableend\end{savenotes}

\sphinxstepscope


\subsubsection{10 Rules for working with geometry}
\label{\detokenize{tutorial/grashopper/documentation_rst/20_10_Tips:rules-for-working-with-geometry}}\label{\detokenize{tutorial/grashopper/documentation_rst/20_10_Tips::doc}}
\sphinxAtStartPar
Following these rules should prevent you from running into errors and performance issues.
This becomes more important in the later phases of a project.


\begin{savenotes}\sphinxattablestart
\sphinxthistablewithglobalstyle
\centering
\begin{tabulary}{\linewidth}[t]{TTT}
\sphinxtoprule
\sphinxtableatstartofbodyhook
\sphinxAtStartPar
1
&
\sphinxAtStartPar
You shall use mesh count wisely
&
\sphinxAtStartPar
Do not use thousands of vertices on a small door handle Manual: Chapter 4.3 PublishGeometry
\\
\sphinxhline
\sphinxAtStartPar
2
&
\sphinxAtStartPar
You shall not use hidden or duplicate objects and materials
&
\sphinxAtStartPar
Purge objects that are never going to be seen by anybody or used by anything
\\
\sphinxhline
\sphinxAtStartPar
3
&
\sphinxAtStartPar
You shall apply level of detail relative to object size, importance and distance
&
\sphinxAtStartPar
Do not spend time and performance doing high level of detail on objects you will never get close to
\\
\sphinxhline
\sphinxAtStartPar
4
&
\sphinxAtStartPar
You shall keep texture resolutions low and relative object sizes
&
\sphinxAtStartPar
Do not use large texture resolution on small objects
\\
\sphinxhline
\sphinxAtStartPar
5
&
\sphinxAtStartPar
You shall trim and subsample point clouds relative to distance and visibility
&
\sphinxAtStartPar
Do not use millions on points on something that is seen from far away or obscured by other objects
\\
\sphinxhline
\sphinxAtStartPar
6
&
\sphinxAtStartPar
You shall only apply collision to objects that  is required to collide with
&
\sphinxAtStartPar
Do not put collision on screws, nails, fixtures etc. Manual: Chapter 4.3 PublishGeometry
\\
\sphinxhline
\sphinxAtStartPar
7
&
\sphinxAtStartPar
You shall target content to specific viewer platforms
&
\sphinxAtStartPar
Do not expect to run a heavy scene on and underpowered platform like a mobile phone or Oculus Standalone
\\
\sphinxhline
\sphinxAtStartPar
8
&
\sphinxAtStartPar
You shall watch the scene for places of performance degradation
&
\sphinxAtStartPar
Always test the scene for places where the performance drops (no lower than 25\sphinxhyphen{}30 fps) and react accordingly with any of the above
\\
\sphinxhline
\sphinxAtStartPar
9
&
\sphinxAtStartPar
You shall only publish when needed
&
\sphinxAtStartPar
Do not spam with content. Use a data dam to control when to send. Use a component for volatile content and another for large static content like context.
\\
\sphinxhline
\sphinxAtStartPar
10
&
\sphinxAtStartPar
You shall report bugs
&
\sphinxAtStartPar
Always report a bug to the radii slack channel
\\
\sphinxbottomrule
\end{tabulary}
\sphinxtableafterendhook\par
\sphinxattableend\end{savenotes}

\sphinxstepscope


\section{Tutorials}
\label{\detokenize{tutorial/Quick_Guide/Quick_Guides:tutorials}}\label{\detokenize{tutorial/Quick_Guide/Quick_Guides::doc}}
\sphinxAtStartPar
Welcome to the guide page, the following resources should give a quick entry into Radii.
The text files will be updated more frequently than the videos, should you come across discrepancies between the videos and what you encounter in Radii.

\sphinxAtStartPar
\sphinxstylestrong{Text Guides}

\sphinxstepscope


\subsection{LV Exploration tutorial}
\label{\detokenize{tutorial/Quick_Guide/1_LV_Exploration:lv-exploration-tutorial}}\label{\detokenize{tutorial/Quick_Guide/1_LV_Exploration::doc}}
\sphinxAtStartPar
This tutorial will introduce the basics of the Radii Grashopper plugin and the Radii Viewer, on the basis of the
introduction given at the Immersive Design Studio at ETH D\sphinxhyphen{}Arch.

\sphinxstepscope


\subsection{LV The simple review tutorial}
\label{\detokenize{tutorial/Quick_Guide/2_LV_Simple_Review:lv-the-simple-review-tutorial}}\label{\detokenize{tutorial/Quick_Guide/2_LV_Simple_Review::doc}}
\sphinxAtStartPar
This tutorial explains what you need for a simple review/discussion with Radii.
It details the usage of views and sections with the PublishViews and PublishSection components.
\begin{itemize}
\item {} 
\sphinxAtStartPar
how to save views in rhino

\item {} 
\sphinxAtStartPar
import them in grashopper

\item {} 
\sphinxAtStartPar
set them through publish view

\item {} 
\sphinxAtStartPar
setting a list of planes \textendash{}\textgreater{} importing them \textendash{}\textgreater{} cycle through

\item {} 
\sphinxAtStartPar
Viewer how to click on views \textendash{}\textgreater{} how to toggle them

\end{itemize}

\sphinxstepscope


\subsection{Lv Presentation \& Story tutorial}
\label{\detokenize{tutorial/Quick_Guide/3_LV Presentation_Storry:lv-presentation-story-tutorial}}\label{\detokenize{tutorial/Quick_Guide/3_LV Presentation_Storry::doc}}
\sphinxAtStartPar
A presentation in Radii is different from showing plans with strictly defined views and perspectives, the participants can
explore for themselves and you can guide them through your model. This explorative approach
is more akin to a museum tour, the visit to an existing building or a performance were you are free to explore for yourself and change perspective.

\sphinxAtStartPar
Thus a presentation in Radii can have specific shared elements, this can be done through
the usage of the PublishAnimation component and the Scenario Manager of the PublishControl component.
With PublishAnimation it is possible to send viewers on defined paths and save a series of controlled moments and interactions with
PublishControl.
\begin{itemize}
\item {} 
\sphinxAtStartPar
walkthrough of how to do an animation

\item {} 
\sphinxAtStartPar
walkthrough on how to work with publishcontrol

\end{itemize}

\sphinxstepscope


\subsection{Lv Collaborative tutorial}
\label{\detokenize{tutorial/Quick_Guide/4_LV_Collaboration:lv-collaborative-tutorial}}\label{\detokenize{tutorial/Quick_Guide/4_LV_Collaboration::doc}}
\sphinxAtStartPar
Usually Radii works in one direction one or many send content to one or a number of viewers, without any feedback.
The collaborative components under the Subscribe section in Rhino can change this and feed back into the Rhino session.
This tutorial will explain possible uses and examples for using them.

\sphinxAtStartPar
Content could be:
\begin{itemize}
\item {} 
\sphinxAtStartPar
easier animation with subscribe to user

\item {} 
\sphinxAtStartPar
users play with geometry collectively and this is then saved as a rhino session

\item {} 
\end{itemize}

\sphinxstepscope


\subsection{Design project documentation}
\label{\detokenize{tutorial/Quick_Guide/5_Documentation:design-project-documentation}}\label{\detokenize{tutorial/Quick_Guide/5_Documentation::doc}}
\sphinxAtStartPar
How a Radii project can be documented for future use:
\begin{itemize}
\item {} 
\sphinxAtStartPar
save everything connected as a radii file to be used locally

\item {} 
\sphinxAtStartPar
save everything to the cloud

\item {} 
\sphinxAtStartPar
save the controlls in a grashopper component

\item {} 
\sphinxAtStartPar
make a recording of a presentation

\end{itemize}

\sphinxAtStartPar
\sphinxstylestrong{Video Guides}

\sphinxAtStartPar
\sphinxhref{https://youtube.com/playlist?list=PLsnxnhDZs-abvOYTdO6MPhOkPDq6ffQNz\&feature=shared}{Thomas Lee \sphinxhyphen{} Radii tutorials playlist}
\begin{itemize}
\item {} 
\sphinxAtStartPar
\sphinxhref{https://www.youtube.com/watch?v=Efk5rdFeWIA\&list=PLsnxnhDZs-abvOYTdO6MPhOkPDq6ffQNz\&index=1\&pp=iAQB}{Basics Publish}

\item {} 
\sphinxAtStartPar
\sphinxhref{https://www.youtube.com/watch?v=Z-RWySQ8r7c\&list=PLsnxnhDZs-abvOYTdO6MPhOkPDq6ffQNz\&index=2\&pp=iAQB}{Basics Save}

\item {} 
\sphinxAtStartPar
\sphinxhref{https://www.youtube.com/watch?v=QaJAsMF3870\&list=PLsnxnhDZs-abvOYTdO6MPhOkPDq6ffQNz\&index=3\&pp=iAQB}{Basics Subscribe}

\item {} 
\sphinxAtStartPar
\sphinxhref{https://www.youtube.com/watch?v=6Lra3IDARNo\&list=PLsnxnhDZs-abvOYTdO6MPhOkPDq6ffQNz\&index=4\&pp=iAQB}{Basics Viewer}

\end{itemize}


\chapter{Indices and tables}
\label{\detokenize{index:indices-and-tables}}\begin{itemize}
\item {} 
\sphinxAtStartPar
\DUrole{xref,std,std-ref}{genindex}

\item {} 
\sphinxAtStartPar
\DUrole{xref,std,std-ref}{search}

\end{itemize}



\renewcommand{\indexname}{Index}
\printindex
\end{document}